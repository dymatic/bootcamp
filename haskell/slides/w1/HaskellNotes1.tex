\documentclass{beamer}

\usepackage{listings}
\lstset{language=Haskell}

\begin{document}
\title{Introduction to Functional Programming with Haskell - Part 1}
\author{Norton Jenkins}
\date{\today}
\frame{\titlepage}

\frame{\frametitle{Table of Contents}\tableofcontents}

\section{Why Haskell?}
\frame{\frametitle{Why haskell?}
  \begin{itemize}
  \item Fully-featured static type system with inference.\pause
    \item Lazy Evaluation (sometimes a burden!)\pause
    \item Interactive development\pause
    \item Large ecosystem\pause
    \item Growing Popularity\pause
    \item High-level abstractions (e.g list comprehensions)\pause
    \item Merits of functional programming (impure vs pure),(mutable vs immutable)
  \end{itemize}
}

\section{Installing Haskell}
\subsection{On Linux}
\frame{\frametitle{Installing Haskell on Linux}
 The package used to run, make, and develop Haskell programs is ``ghc''. Installing ghc from your package manager should be enough.
  \begin{description}
  \item[Ubuntu/Debian] \lstinline!sudo apt-get install ghc!
  \item[Fedora/RHEL] \lstinline!sudo yum install ghc!
  \item[Opensuse] \lstinline!sudo zypper install ghc!
  \item[Gentoo] \lstinline!sudo emerge ghc!
  \item[Arch] \lstinline!sudo pacman -S ghc!
  \end{description}
}

\subsection{On Windows}
\frame{\frametitle{Installing Haskell on Windows}
You can install the entire Haskell platform on Windows from \href{http://www.haskell.org/platform/}{The Haskell Platform Website}. This will install tools like Cabal, which must be installed seperately through most package managers on Linux.
}

\section{ghci}
\frame{\frametitle{ghci - Interactive Development}
\begin{itemize}
\item Interprets Haskell statements on-the-fly.\pause
\item Similar to ``clisp'' for Lisp, ``python'' for Python, ``lein repl'' for Clojure, etc.\pause
\item Advanced capabilities such as loading sources and debugging built in.\pause
\item Easy way to type-check using type inference, or test one function to make sure it works.
\end{itemize}
}
\subsection{Hello World}
\frame{\frametitle{Hello Worlds in Haskell}
\begin{itemize}
\item 2 implementations, 1 pure one impure.\pause
\item impure in {\bf main} function with do block.
\end{itemize}
}
\end{document}
